% Created 2024-10-24 Thu 19:09
% Intended LaTeX compiler: pdflatex
\documentclass[11pt]{article}
\usepackage[utf8]{inputenc}
\usepackage[T1]{fontenc}
\usepackage{graphicx}
\usepackage{longtable}
\usepackage{wrapfig}
\usepackage{rotating}
\usepackage[normalem]{ulem}
\usepackage{amsmath}
\usepackage{amssymb}
\usepackage{capt-of}
\usepackage{hyperref}
\usepackage{color}
\usepackage{tikz}
\author{Błażej Niewiadomski}
\date{\today}
\title{{[}02.10.2024] - wstęp do liczb zespolonych}
\hypersetup{
 pdfauthor={Błażej Niewiadomski},
 pdftitle={{[}02.10.2024] - wstęp do liczb zespolonych},
 pdfkeywords={},
 pdfsubject={},
 pdfcreator={Emacs 29.4 (Org mode 9.8-pre)}, 
 pdflang={English}}
\begin{document}

\maketitle
\tableofcontents

\section*{Grupy}
\label{sec:org12ffa52}

\begin{description}
\item[{grupa}] zbiór G z działaniem \^{}, gdy spełnione są warunki:

\begin{enumerate}
\item działanie \^{} jest \emph{łączne} tzn. \(\forall\) a, b, c ∈ G : a \^{} (b \^{} c) \(\equiv\) (a \^{} b) \^{} c
\item istnieje element neutralny e działania \^{}, czyli taki, że \(\forall\) a ∈ G : a \^{} e \(\equiv\) e \^{} a \(\equiv\) a
\item \(\exists\)\textsubscript{a \(\in\) G} a\textsuperscript{-1} \(\in\) G : a \^{} a\textsuperscript{-1} \(\equiv\) 1
\end{enumerate}
\end{description}

\{
\begin{equation}
\left[ \frac{t^{2}}{2} \right]
\end{equation}
\end{document}
